\documentclass{article}

% set font encoding for PDFLaTeX or XeLaTeX
\usepackage{ifxetex}
\ifxetex
\usepackage{fontspec}
\else
\usepackage[T1]{fontenc}
\usepackage[utf8]{inputenc}
\usepackage{amsmath}
\usepackage{lmodern}
\usepackage{textcomp}
\usepackage{amsfonts}
\usepackage{amssymb}
\usepackage{graphicx}
\usepackage[left=2cm,right=2cm,top=2cm,bottom=2cm]{geometry}

% used in maketitle
\title{Evaluacion 1}
\author{Eduardo Castillo Bastida\\
Departamento de Fisica \\
Universidad de Sonora}
\date{30 de octubre de 2017}


% Enable SageTeX to run SageMath code right inside this LaTeX file.
% documentation: http://mirrors.ctan.org/macros/latex/contrib/sagetex/sagetexpackage.pdf
% \usepackage{sagetex}

\begin{document}
\maketitle
\clearpage

\section{Actividad 1: Esfera}
\begin{verbatim}
program Sphere

! Calcular el volumen de una esfera.
!
! Declarar las variables.
! Constantes pi
! variables= radio al cuadrado y altura

  implicit none    

  integer :: ierr
  character(1) :: yn
  real :: radius, area, vol
  real, parameter :: pi = 3.141592653589793

  interactive_loop: do

!   Le pediremos al usuario por el valor del radio
!   y las leeremos.

    write (*,*) 'Ingrese por favor el radio a utilizar'
    read (*,*,iostat=ierr) radius

!   If radius and height could not be read from input,
!   then cycle through the loop.

    if (ierr /= 0) then
      write(*,*) 'Error, invalid input.'
      cycle interactive_loop
    end if

!   Compute area.  The ** means "raise to a power."

    area = 4*pi * (radius**2)

    vol=(4/3)*pi * (radius**3)

!   Write the input variables (radius, height)
!   and output (area) to the screen.

    write (*,'(1x,a7,f14.2,5x,a7,f14.2,5x,a9,f14.2)') &
         'radius=',radius,'area=',area
    
    write (*,'(1x,a7,f14.2,5x,a7,f14.2,5x,a9,f14.2)') &
         'radius=',radius, 'volume=',vol
        
    

    yn = ' '
    yn_loop: do
      write(*,*) 'Perform another calculation? y[n]'
      read(*,'(a1)') yn
      if (yn=='y' .or. yn=='Y') exit yn_loop
      if (yn=='n' .or. yn=='N' .or. yn==' ') exit interactive_loop
    end do yn_loop

 end do interactive_loop

 

end program Sphere
\end{verbatim}
\begin{figure}[htbp]
\centering
\includegraphics[width=12cm]{Esfera.png}
\caption{Resultados.}\label{fig:figura1}
\end{figure}

\section{Actividad 2: Medias}
\begin{verbatim}
program summation
implicit none
integer :: sum, a, count
real :: arit, harm, sumainv
real :: fa, fc, fs

print*, "Este programa realiza las medias de una sumatoria,"
print*, "cuando quiera aplaste 0 para terminar"
open(unit=10, file="SumData.DAT", status='unknown')

suma = 0
count = 0
sumainv = 0

do
 print*, "Add:"
 read*, a
 if (a == 0) then
  exit
 else
sum = sum + a
count = count + 1
fa = float(a)
fa = 1/fa
sumainv = sumainv + fa

 end if
 write(10,*) a
end do
fs = float(sum)
fc = float(count)
arit = fs / fc
harm = fc / sumainv


print*, "Sumatoria =", sum
write(10,*) "Sumatoria =", sum
write(10,*)' '
print*, "Media aritmetica =", arit
write(10,*) "Media aritmetica =", arit
write(10,*) ' '
print*, "Media armonica =", harm
write(10,*) "Media armonica =", harm
write(10,*) ' '


close(10)

end
\end{verbatim}
\begin{figure}[htbp]
\centering
\includegraphics[width=12cm]{Media.png}
\caption{Resultados.}\label{fig:figura1}
\end{figure}

\section{Actividad 3: Leibniz}
\begin{verbatim}
Program Liebniz
  ! Este programa calcula el valores del numero pi usando la serie de
  ! Leibniz
  ! declaracion de variables
  implicit none
  integer:: i, n  
  real :: pi, serie

  ! Pregunta por el numero de terminos de la serie
  write(*,*) 'Escribe el valor de n, número de terminos de la serie'
  read (*,*) n

  serie = 0.
  
  do i=0,n

     serie = serie + (((-1)**i)/(2*real(i) + 1))

  end do
  
pi= serie * 4.

  write (*,*) 'cuando n =',n, 'pi=',pi
end program 


\end{verbatim}
\begin{figure}[htbp]
\centering
\includegraphics[width=12cm]{Pi.png}
\caption{Resultados.}\label{fig:figura1}
\end{figure}



\end{document}
\documentclass{article}

% set font encoding for PDFLaTeX or XeLaTeX
\usepackage{ifxetex}
\ifxetex
  \usepackage{fontspec}
\else
  \usepackage[T1]{fontenc}
  \usepackage[utf8]{inputenc}
  \usepackage{lmodern}
\fi

% used in maketitle
\title{Bitácora}
\author{Eduardo Castillo Bastida}

% Enable SageTeX to run SageMath code right inside this LaTeX file.
% documentation: http://mirrors.ctan.org/macros/latex/contrib/sagetex/sagetexpackage.pdf
% \usepackage{sagetex}

\begin{document}
\maketitle
1-.Command Prompt
	batman@ltsp28:
		Nos abre una terminal en la cual se pueden escribir y ejecutar
		distintos comandos.
		Dentro de éste, podemos observar tambien (al ejecutar un comando) que
		se producen ciertas líneas que indican la acción correspondiente.
		Al terminar de indicar lo ejecutado, nos devuelve a nuestra linea de
		comando ejecutable
        
2-. Basic Navigation.
	Comando: pwd.
		Nos indica en dónde estamos actualmente, es decir, en qué directorio.
	Comando: ls.
		Nos muestra los contenidos del "lugar" donde nos encontramos
	Comando: ls-1
		Muestra una lista larga de los contenidos de nuestra ubicacion, dicha
		lista contiene lo siguiente:
		*El primer caracter indica si es un archivo normal (-) o si es un
		directorio (d)
		*Los siguientes 9 carcteres son permisos para el archivo o directorio
		*El campo siguiente es el número de bloques
		*El siguiente campo es el dueño del archivo o directorio
		*Eñ siguiente campo es el grupo de archivos o directorios al cual
		pertenece
		*Siguiendo con el tamaño del archivo
		*Sigue con la última modificación
		*Finalmente tenemos el nombre del archivo
       
3-. More About Files!
	Comando:file.exe
		Un archivo ejecutable o programa
	Comando: file.txt
		Un archivo simple de texto
	Comando: file.png
		Una imagen

4-.Manual Pages!
	Comando: man
		Es un comando de búsqueda y descripción.
		Nos dice el nombre del comando y una pequeña descripción
	Comando: man -k
		Realiza una búsqueda en base a una palabra clave

5-. File Manipulation
	Comando: mkdir
		Con éste comando podremos crear un directorio
	Comando: mkdir -p
		Con -p crearemos un directorio "padre" como sea necesario
	COmando: mkdir -v
		Con -v hace que el comando mkdir nos diga que es lo que está haciendo en este 			momento.

6-. Vi Text Editor!
	Existen 2 maneras de VI (Insertar y Editar):
		Insertar hace que insertes un contenido en un archivo
		Editar hace que edites el contenido ya creado anteriormente dentro de un archivo
			Comando: vi <file>

7-. Wildcards!
	Esto, es un set para crear bloques que te permiten crear un patrón definiendo un grupo de 	archivos o directorios
		*- Representa cero o más caracteres
		?- Representa un único caracter
		[]-Representa un rango de caracteres
Nota: El comando ls b* nos indicará que hemos utilizado un patrón de archivos o directorios, lo cuál el mismo programa traducirá a lo que nos queremos referir

8-. Permisos!
	Los permisos especifican a una persona en particular que puede o no para que haga o que no haga con un archivo dentro de tu programa.
		Commando: r
			Puede ver los contenidos de este archivo
		Commando: w
			Puede cambiar los contenidos del archivo
		Commando: x
			Puede ejecutar o correr archivos que estén en el programa
	Para cada archivo definiremos 3 grupos de personas a quien le daremos permisos
		Commando: owner
	|		Persona que creó el archivo
		Commando: group
			Todo archivo pertenece a un grupo de personas
		Commando: others
			Todos quienes no sean parte del grupo o el dueño

\end{document}

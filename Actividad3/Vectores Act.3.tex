\documentclass[12pt,a4paper]{article}

% set font encoding for PDFLaTeX or XeLaTeX
\usepackage{ifxetex}
\ifxetex
  \usepackage{fontspec}
\else
  \usepackage[T1]{fontenc}
  \usepackage[utf8]{inputenc}
  \usepackage[spanish]{babel}
  \usepackage{lmodern}
  \usepackage{amsmath}
  \usepackage{amsfonts}
  \usepackage{amssymb}
  \usepackage{graphicx}
  \usepackage[left=2cm,right=2cm,top=2cm,bottom=2cm]{geometry}
\fi

% used in maketitle
\title{\textbf{Movimiento de Proyectiles Act.3}}
\author{Eduardo Castillo Bastida}

% Enable SageTeX to run SageMath code right inside this LaTeX file.
% documentation: http://mirrors.ctan.org/macros/latex/contrib/sagetex/sagetexpackage.pdf
%\usepackage{graphicx}


\begin{document}
\maketitle
El movimiento de un proyectil es un tipo de movimiento en el cual un objeto o partícula describe una trayectoria parabólica. La posición en cualquier instante durante su recorrido, se puede determinar según las siguientes ecuaciones:
\begin{eqnarray}
x=v_{0}tcos(\theta)\nonumber\\
y=v_{0}tsin(\theta)-gt^{2}
\end{eqnarray}
El tipo de trayectoria, el alcance horizontal y la altura máxima del movimiento de un proyectil en cercanías de la superficie terrestre, dependen de los parámetros iniciales del ángulo y la velocidad inicial del lanzamiento.
\section{Dependencia del ángulo}
La siguiente actividad consiste en determinar las gráficas de las trayectorias de un proyectil en función del ángulo de lanzamiento. Se requiere especificar la velocidad inicial y el número de puntos de cada gráfica. Los ángulos tomados corresponden a $15^{o}$, $30^{o}, 45^{o}, 60^{o}, 75^{o}$ y $90^{o}$. El número de puntos para cada gráfica  corresponde a 20. El tiempo de vuelo será calculado con  la siguiente ecuación:
\begin{eqnarray}
t=\frac{2v_{0}sin(\theta)}{g}
\end{eqnarray}
Si se selecciona, por ejemplo, una velocidad inicial de 10 m/s. Las trayectorias obtenidas para los ángulos seleccionados, corresponden a los mostrados en la figura \ref{fig:figura1}:
\begin{figure}
\centering
\includegraphics[width=8cm]{GraficaVectores.png}
\caption{Trayectoria de un proyectil en función del ángulo ($\theta$)}\label{fig:figura1}
\end{figure}
\section{Aplicacíon Fortran}
La aplicación Fortran empleado para obtener el conjunto de datos para los ángulos especificados, corresponde a:
\begin{verbatim}
program Vector
    implicit none

  ! definimos constantes
    real, parameter :: g = 9.81
    real, parameter :: pi = 3.1415927
  
  ! definimos las variables
    integer::i, k, nps
    real :: a, vi, tv
    real, dimension(20):: t=0.,x=0., y=0.
  

  ! Leer valores para el ángulo a, y la velocidad inicial u desde la terminal
  
    write(*,*) 'MOVIMIENTO PARABÓLICO'
    
    write(*,*) 'DATOS DESPLAZAMIENTO VERTICAL Y HORIZONTAL'
    
    write(*,*) 'Introduzca los valores de  la velocidad inicial (vi) en m&
       &/s y el número de datos'
  
    read(*,*) vi,nps

  !Definimos el ciclo
    cicloenangulo:do k=15,90,15
      ! convirtiendo ángulo a radianes
       a = k * pi / 180.0

      ! las ecuación para el cálculo del tiempo de vuelo
       tv=2*vi*sin(a)/g
       
  ciclodeposicion: do i=0,nps
     t(i)=t(i)+i*(tv/real(nps))
     x(i)=x(i)+vi*t(i)*cos(a)
     y(i)=y(i)+(2*vi*t(i)*sin(a)-g*t(i)**2)/2
  ! output data to a file
     open(1, file='datos.dat', status='unknown')
     write(1,1000) x(i), y(i)
     1000 format(f15.10,5x,f15.10)
  end do ciclodeposicion
  write(1,1100)
  1100 format(/)
 
  do i=0,nps
     t(i)=0
     x(i)=0
     y(i)=0
     end do
  end do cicloenangulo
   close(1)

\end{verbatim}

\end{document}